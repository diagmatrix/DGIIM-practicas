% PLANTILLA PARA TRABAJOS EN LATEX
% Autor: Manuel Gachs Ballegeer
% Github: https://github.com/Manuelbelgicano
% Licencia: GNU General Public License v3.0
% Versión: 0.2

\documentclass[12pt,titlepage,a4paper]{article}

%%%%%%%%%%%%%%%%%%%%%%%%%%%%%%%%%%%%%%%%%
%			   COLORINES				%
%%%%%%%%%%%%%%%%%%%%%%%%%%%%%%%%%%%%%%%%%
\usepackage{xcolor}
% COLORES DE LA ESTRUCTURA DEL DOCUMENTO
\definecolor{ed_1}{HTML}{8A0808} % Portada
\definecolor{ed_2}{HTML}{FFFFFF} % Texto de la portada
\definecolor{ed_3}{HTML}{8A0808} % Títulos de las secciones
\definecolor{ed_4}{HTML}{610505} % Títulos de las subsecciones
\definecolor{ed_5}{HTML}{300303} % Títulos de las subsubsecciones
\definecolor{ed_6}{HTML}{610B0B} % Texto del encabezado
% COLORES PARA MATEMÁTICAS
\definecolor{m_1}{HTML}{8A0808} % Teoremas, lemas
\definecolor{m_2}{HTML}{610505} % Definiciones
\definecolor{m_3}{HTML}{300303} % Corolarios
\definecolor{m_4}{HTML}{000000} % Ejemplos, demostraciones
% COLORES PARA CÓDIGO
\definecolor{c_1}{HTML}{8A0808} % Línea a la izquierda
\definecolor{c_2}{HTML}{610B0B} % Palabras reservadas
\definecolor{c_3}{HTML}{B40431} % Cadenas de caracteres
\definecolor{c_4}{HTML}{333333} % Comentarios
% COLORES PARA LISTAS
\definecolor{l_1}{HTML}{8A0808} % Primer símbolo
\definecolor{l_2}{HTML}{610505} % Primera indentación
\definecolor{l_3}{HTML}{300303} % Segunda indentación
\definecolor{l_4}{HTML}{000000} % Tercera indentación
% COLORES PARA MARCOS
\definecolor{ma_1}{HTML}{8A0808} % Línea marco ejemplos
\definecolor{ma_2}{HTML}{610505} % Línea marco notas
\definecolor{ma_3}{HTML}{FFEDEE} % Fondo ejemplos
\definecolor{ma_4}{HTML}{FFE0DF} % Fondo notas

%%%%%%%%%%%%%%%%%%%%%%%%%%%%%%%%%%%%%%%%%
%				  MARCOS				%
%%%%%%%%%%%%%%%%%%%%%%%%%%%%%%%%%%%%%%%%%
\usepackage[xcolor]{mdframed} % Marcos
\mdfsetup{ % Configuración general de los marcos
  skipabove=1em, % Espacio sobre los marcos
  skipbelow=1em, % Espacio bajo los marcos
  innertopmargin=0.3em, % Margen interior superior
  innerbottommargin=1em, % Margen interior inferior
  splittopskip=2\topsep, % Espacio entre marcos
  usetwoside=false, % Diferenciar páginas
}

\mdfdefinestyle{marco_ejemplos}{ % Nombre del estilo
	linewidth=2pt, % Grosor de la línea
	linecolor=ma_1, % Color de la línea
	backgroundcolor=ma_3, % Color de fondo
	topline=false, % Línea arriba
	leftline=true, % Línea a la izquierda
	bottomline=false, % Línea abajo
	rightline=false,% Línea a la derecha
	leftmargin=0em, % Margen a la izquierda
	innerleftmargin=1em, % Margen interior a la izquierda
	innerrightmargin=1em, % Margen interior a la derecha
	rightmargin=0em, % Margen a la derecha
}
\mdfdefinestyle{marco_anotaciones}{ % Nombre del estilo
	linewidth=1pt, % Grosor de la línea
	linecolor=ma_2, % Color de la línea
	backgroundcolor=ma_4, % Color de fondo
	topline=true, % Línea arriba
	leftline=true, % Línea a la izquierda
	bottomline=true, % Línea abajo
	rightline=true,% Línea a la derecha
	leftmargin=0em, % Margen a la izquierda
	innerleftmargin=1em, % Margen interior a la izquierda
	innerrightmargin=1em, % Margen interior a la derecha
	rightmargin=0em, % Margen a la derecha
}

%%%%%%%%%%%%%%%%%%%%%%%%%%%%%%%%%%%%%%%%%
%			  MATEMÁTICAS				%
%%%%%%%%%%%%%%%%%%%%%%%%%%%%%%%%%%%%%%%%%
\usepackage{amsmath} % Matemáticas
\usepackage{amsfonts} % Letras caligráficas para matemáticas
\usepackage{mathtools} % Matemáticas extra
\usepackage{amsthm} % Teoremas
\renewenvironment{proof}{{\bfseries\color{m_4}Demostración:}}{\qed} % Cambiar el título de las demostraciones
% ESTILOS DE TEOREMAS
\newtheoremstyle{teorema} % Nombre del estilo
{} % Espacio por encima
{} % Espacio por debajo
{} % Estilo del cuerpo
{} % Indentación
{\color{m_1}\bfseries} % Estilo de la cabecera
{:} % Símbolo tras la cabecera
{ } % Espacio tras la cabecera
{} % Especificación de la cabecera
\newtheoremstyle{lema} % Nombre del estilo
{} % Espacio por encima
{} % Espacio por debajo
{} % Estilo del cuerpo
{} % Indentación
{\color{m_1}\bfseries} % Estilo de la cabecera
{:} % Símbolo tras la cabecera
{ } % Espacio tras la cabecera
{} % Especificación de la cabecera
\newtheoremstyle{definicion} % Nombre del estilo
{} % Espacio por encima
{} % Espacio por debajo
{} % Estilo del cuerpo
{} % Indentación
{\color{m_2}\bfseries} % Estilo de la cabecera
{:} % Símbolo tras la cabecera
{ } % Espacio tras la cabecera
{} % Especificación de la cabecera
\newtheoremstyle{corolario} % Nombre del estilo
{} % Espacio por encima
{} % Espacio por debajo
{} % Estilo del cuerpo
{} % Indentación
{\color{m_3}\bfseries} % Estilo de la cabecera
{:} % Símbolo tras la cabecera
{ } % Espacio tras la cabecera
{} % Especificación de la cabecera
\newtheoremstyle{nota} % Nombre del estilo
{} % Espacio por encima
{} % Espacio por debajo
{\ttfamily} % Estilo del cuerpo
{} % Indentación
{\color{m_4}\bfseries} % Estilo de la cabecera
{:} % Símbolo tras la cabecera
{ } % Espacio tras la cabecera
{} % Especificación de la cabecera
\newtheoremstyle{ejemplo} % Nombre del estilo
{} % Espacio por encima
{} % Espacio por debajo
{} % Estilo del cuerpo
{} % Indentación
{\color{m_4}\itshape\bfseries} % Estilo de la cabecera
{:} % Símbolo tras la cabecera
{ } % Espacio tras la cabecera
{} % Especificación de la cabecera
% COMANDOS
% Eliminar '*' para añadir numeración a los lemas, teoremas...
\theoremstyle{definicion}
\newtheorem*{defi}{Definición} % Comando para las definiciones
\theoremstyle{lema}
\newtheorem*{lem}{Lema} % Comando para los lemas
\theoremstyle{teorema}
\newtheorem*{teo}{Teorema} % Comando para los teoremas
\theoremstyle{corolario}
\newtheorem*{cor}{Corolario} % Comando para los corolarios
\theoremstyle{ejemplo}
\newtheorem*{eje}{Ejemplo} % Comando para los ejemplos
\theoremstyle{nota}
\newtheorem*{ano}{Acciones} % Comando para las anotaciones
\surroundwithmdframed[style=marco_ejemplos]{eje}
\surroundwithmdframed[style=marco_anotaciones]{ano}

%%%%%%%%%%%%%%%%%%%%%%%%%%%%%%%%%%%%%%%%%
%			   TIPOGRAFÍA				%
%%%%%%%%%%%%%%%%%%%%%%%%%%%%%%%%%%%%%%%%%
\usepackage{CrimsonPro}
\usepackage[T1]{fontenc}
% The font package uses mweights.sty which has som issues with the
% \normalfont command. The following two lines fixes this issue.
\let\oldnormalfont\normalfont
\def\normalfont{\oldnormalfont\mdseries}

%%%%%%%%%%%%%%%%%%%%%%%%%%%%%%%%%%%%%%%%%
%				  CÓDIGO				%
%%%%%%%%%%%%%%%%%%%%%%%%%%%%%%%%%%%%%%%%%
\usepackage{listingsutf8}
\lstset{
	inputencoding=utf8/latin1, % Codificación
	xleftmargin=1em, % Margen extra a la izquierda
	breaklines=true, % Romper líneas largas
	language=C, % Lenguaje del código
	frame=none, % Enmarcado
	numbers=none, % Números de línea
	numbersep=8pt, % Separación de los números de línea
	tabsize=4, % Tamaño de los tabs
	frame=none, % Posición del enmarcado
	framerule=2pt, % Grosor del enmarcado
	showstringspaces=false, % Mostrar los espacios en las cadenas de caracteres
	basicstyle=\footnotesize\ttfamily, % Estilo del código
	keywordstyle=\color{c_2}, % Estilo de las palabras reservadas
	numberstyle=\normalfont, % Estilo de los números de línea
	rulecolor=\color{c_1}, % Estilo del enmarcado
	commentstyle=\color{c_4}, % Estilo de los comentarios
	stringstyle=\color{c_3} % Estilo de las cadenas de caracteres
}
\surroundwithmdframed[style=marco_ejemplos]{lstlisting}
%%%%%%%%%%%%%%%%%%%%%%%%%%%%%%%%%%%%%%%%%
%				MÁRGENES				%
%%%%%%%%%%%%%%%%%%%%%%%%%%%%%%%%%%%%%%%%%
\usepackage[a4paper]{geometry}
\geometry{
	left=2.5cm, % Margen izquierdo
	right=2.5cm, % Margen derecho
	bottom=2.5cm % Margen inferior
}

%%%%%%%%%%%%%%%%%%%%%%%%%%%%%%%%%%%%%%%%%
%  			  LISTAS/TABLAS				%
%%%%%%%%%%%%%%%%%%%%%%%%%%%%%%%%%%%%%%%%%
\usepackage{enumitem} % Opciones de personalización de listas
\renewcommand{\arraystretch}{1.3} %Cambiar el tamaño entre líneas de una tabla
% SÍMBOLOS LISTAS
\renewcommand{\labelitemi}{\color{l_1}$\bullet$} % Primer símbolo
\renewcommand{\labelitemii}{\color{l_2}$\circ$} % Símbolo primera indentación
\renewcommand{\labelitemiii}{\color{l_3}$\diamond$} % Símbolo segunda indentación
\renewcommand{\labelitemiv}{\color{l_4}$-$} % Símbolo tercera indentación
% SÍMBOLOS ENUMERACIONES
\renewcommand{\labelenumi}{\color{l_1}\bfseries\arabic{enumi}.} % Primer símbolo
\renewcommand{\labelenumii}{\color{l_2}\bfseries\Roman{enumii}.} % Símbolo primera indentación
\renewcommand{\labelenumiii}{\color{l_3}\bfseries(\alph{enumiii})} % Símbolo segunda indentación
\renewcommand{\labelenumiv}{\color{l_4}\bfseries\Alph{enumiv}.} % Símbolo tercera indentación
% DESCRIPCIONES
\renewcommand{\descriptionlabel}[1]{\hspace{\labelsep}\color{l_1}\textbf{#1}} % Color y estilo del título de la descripción

%%%%%%%%%%%%%%%%%%%%%%%%%%%%%%%%%%%%%%%%%
%		COMANDOS PERSONALIZADOS 		%
%%%%%%%%%%%%%%%%%%%%%%%%%%%%%%%%%%%%%%%%%
\newcommand{\contenta}{ % Insitución
	Universidad de Granada
}
\newcommand{\contentb}{ % Año o cualquier otra información para el título
	Curso 2020-2021
}
\newcommand{\contentc}{ % Autores del documento
	\begin{tabular}{l}
	Manuel Gachs Ballegeer	
	\end{tabular}
}

%%%%%%%%%%%%%%%%%%%%%%%%%%%%%%%%%%%%%%%%%
%		ENCABEZADO/PIE DE PAGINA		%
%%%%%%%%%%%%%%%%%%%%%%%%%%%%%%%%%%%%%%%%%
\usepackage{fancyhdr}
\setlength{\headheight}{14pt}
\pagestyle{fancy}
\fancyhf{}
% Para que aparezca el título de la sección y no el número 
\renewcommand{\sectionmark}[1]{%
\markboth{#1}{}}
% Encabezado
\fancyhead[LE,RO]{\color{ed_6}{\leftmark}} % A la izquierda en pares, derecha en impares
\fancyhead[RE,LO]{\color{ed_6}{Modelos de Computación}} % A la derecha en pares, izquierda en impares
% Pie de página
\fancyfoot[LE,RO]{\Large\textbf{\thepage}} % A la izquierda en pares, derecha en impares
\renewcommand{\headrulewidth}{0.5pt} % Grosor de la línea

%%%%%%%%%%%%%%%%%%%%%%%%%%%%%%%%%%%%%%%%%
%			   	TÍTULOS					%
%%%%%%%%%%%%%%%%%%%%%%%%%%%%%%%%%%%%%%%%%
\usepackage{titlesec}
\titleformat{\section} % Estilo de las secciones
{\color{ed_3}\Huge\bfseries}
{\color{ed_3}\thesection}{1em}{}
\titleformat{\subsection} % Estilo de las subsecciones
{\color{ed_4}\LARGE\bfseries}
{\color{ed_4}\thesubsection}{1em}{}
\titleformat{\subsubsection} % Estilo de las subsecciones
{\color{ed_4}\bfseries}
{\color{ed_4}\thesubsubsection}{1em}{}

%%%%%%%%%%%%%%%%%%%%%%%%%%%%%%%%%%%%%%%%%
%		   	  MISCELÁNEO				%
%%%%%%%%%%%%%%%%%%%%%%%%%%%%%%%%%%%%%%%%%
\usepackage{pagecolor} % Colorear las portadas
\renewcommand{\contentsname}{Índice} % Cambiar el título del índice
\setlength\parindent{0pt} % Tamaño de la sangría
\usepackage{graphicx} % Imágenes
\usepackage{blindtext} % Texto de relleno (Se puede eliminar)

\begin{document}

%%%%%%%%%%%%%%%%%%%%%%%%%%%%%%%%%%%%%%%%%
%				 PORTADA 				%
%%%%%%%%%%%%%%%%%%%%%%%%%%%%%%%%%%%%%%%%%
\begin{titlepage}
	\newpagecolor{ed_1} % Color de la portada
	\parbox[t]{\textwidth}{
		\raggedright
		\color{ed_2}{\LARGE{\textbf{\contenta}}} \\
		\textit{\contentb}
	}
	\vfill
	\parbox[c]{\textwidth}{
		\color{ed_2}{
			\fontsize{70pt}{70pt}{\textbf{Lex/Flex}} \\ % Título
			\bigskip \\
			\fontsize{40pt}{40pt}{\emph{Modelos de Computación}} % Subtítulo
		}
	}
	\vfill
	\parbox[t]{\textwidth}{
		\raggedright
		\color{ed_2}{\Large{\contentc}} \\
	}
\end{titlepage}
\restorepagecolor

%%%%%%%%%%%%%%%%%%%%%%%%%%%%%%%%%%%%%%%%%
%				 ÍNDICE 				%
%%%%%%%%%%%%%%%%%%%%%%%%%%%%%%%%%%%%%%%%%
\tableofcontents
\clearpage

%%%%%%%%%%%%%%%%%%%%%%%%%%%%%%%%%%%%%%%%%
%				 DOCUMENTO 				%
%%%%%%%%%%%%%%%%%%%%%%%%%%%%%%%%%%%%%%%%%

\section{Descripción del problema}

Muchas veces, al realizar copias de fragmentos de artículos de Wikipedia a
documentos de texto, la copia no tiene el formato deseado. Normalmente un
párrafo completo aparece en una única línea, o justo lo contrario, cada
palabra se copia en una nueva línea. Además, los superíndices que señalan a
la bibliografía suelen copiarse como caracteres usuales. Como ejemplo, esto
es lo que ocurre con este pequeño fragmento del artículo de \emph{La Historia
Interminable}, la novela del esritor Michael Ende:
\smallskip
$$\text{[...] diversas adaptaciones cinematográficas.}^2\to
\text{[...] diversas adaptaciones cinematográficas.2}$$
\smallskip
El objetivo del programa es, por tanto, procesar un texto sacado de la web y
transformarlo en un texto usual. Más concretamente, debe realizar las siguientes
acciones:
\begin{enumerate}
\item Limitar el número de caracteres por línea a un valor predeterminado.
\item En líneas con un número de caracteres menor al predeterminado, unir
varias líneas siempre y cuando no pertenezcan a un nuevo párrafo.
\item Eliminar líneas en blanco innecesarias.
\item Eliminar los superíncides de las referencias del texto.
\end{enumerate}
También en ocasiones es interesante conocer el número de palabras que contienen
tilde de un determinado texto, por tanto el programa también debe contar el 
número de palabras distintas que contienen tilde.
\newpage
\section{Solución al problema}

La descripción de la solución se divide en varias secciones. Primeramente se
explican las variables y funciones utilizadas en el programa, luego las reglas
y finalmente el proceso de ejecución.

\subsection{Declaraciones}

La sección de declaraciones del archivo plantilla del programa contiene estas
sentencias:
\smallskip
\begin{lstlisting}
#include <stdio.h>
#include <ctype.h>
#include <string.h>
FILE * salida;
unsigned MAX_CHAR = 80;
unsigned num_char;
unsigned lin_cor;
unsigned ref_elim;
unsigned p_tilde;
char tildes[5000][100];
void escribe_palabra(FILE * fichero,char * datos);
void anade_palabra_tilde(char * palabra);
\end{lstlisting}

Las variables de la sección corresponden con lo siguiente:
\begin{description}[align=right,labelwidth=4em,noitemsep]
\item [\texttt{salida}] Identificador del archivo de salida.
\item [\texttt{MAX\_CHAR}] Número máximo de caracteres por línea.
\item [\texttt{num\_char}] Número de caracteres de la línea actual en el
archivo de salida.
\item [\texttt{lin\_cor}] Número de veces que el programa genera una nueva
línea en el programa de salida.
\item [\texttt{ref\_elim}] Número de referencias eliminadas del texto.
\item [\texttt{p\_tilde}] Número de palabras únicas con tilde.
\item [\texttt{tildes}] Lista de palabras únicas con tilde.
\end{description}
La función \texttt{escribe\_palabra} escribe la palabra dada como argumento
en el archivo de salida dado también como argumento. Su implementación es:
\begin{lstlisting}
void escribe_palabra(FILE * fichero,char * datos) {
    fprintf(fichero,"%s",datos);
}
\end{lstlisting}
La función \texttt{anade\_palabra\_tilde} es más compleja, puesto que ante
una palabra de entrada debe comprobar que esa palabra no se haya encontrado
con tilde anteriormente en el texto, y en caso de ser nueva, añadirla como
palabra con tilde. La implementación de la función es la siguiente:
\begin{lstlisting}
void anade_palabra_tilde(char * palabra) {
    char * copia = strdup(palabra);
    for (int i=0;i<p_tilde;i++) {
        copia[0] = tolower(copia[0]);
        if (strcmp(tildes[i],copia)==0)
            return;
        copia[0] = toupper(copia[0]);
        if (strcmp(tildes[i],copia)==0)
            return;
    }
    sprintf(tildes[p_tilde],"%s",palabra);
    p_tilde++;
}
\end{lstlisting}

\subsection{Reglas}
\smallskip
\subsubsection*{\texttt{([a-zñA-Z])*[áéíóú]([a-z]|ñ)*}}
\begin{lstlisting}
{anade_palabra_tilde(yytext);REJECT;}
\end{lstlisting}
Esta expresión regular detecta las palabras que contienen tilde. El programa,
si una palabra encaja con esta expresión regular, la añade al contador de
palabras que contienen tilde, y busca otras expresiones regulares que encajen
también con esta palabra.

\subsubsection*{\texttt{(,|\textbackslash.)[0-9]+}}
\begin{lstlisting}
{ref_elim++;unput(yytext[0]);}
\end{lstlisting}
Esta expresión regular detecta las referencias copiadas del texto, dejando
sin embargo los signos de puntuación para el procesamiento de otras reglas.

\subsubsection*{\texttt{"[cita requerida]"}}
\begin{lstlisting}
{ref_elim++}
\end{lstlisting}
Esta regla es equivalente a la anterior.
\newpage
\subsubsection*{\texttt{[\textasciicircum\; \textbackslash t\textbackslash 
n\textbackslash.,:]+}}
\begin{lstlisting}
{num_char += yyleng;
char word_buffer[100];
if (num_char<MAX_CHAR) { 
	sprintf(word_buffer,"%s",yytext);
} else { 
	num_char = yyleng;
	lin_cor++;
	sprintf(word_buffer,"\n%s",yytext);
}
escribe_palabra(salida,word_buffer);}
\end{lstlisting}
Esta es la regla más importante del programa. La expresión regular detecta 
todas las palabras del texto, y las escribe en la línea actual del archivo de 
salida, salvo que eso produjese que el número de caracteres de la línea supere 
el límite establecido, en cuyo caso inserta un salto de línea antes de
escribirla.

\subsubsection*{\texttt{(\textbackslash n\textbackslash n)+}}
\begin{lstlisting}
{num_char = 0;escribe_palabra(salida,"\n");}
\end{lstlisting}
Detecta los saltos de línea innecesarios y los transforma en un único salto.

\subsubsection*{\texttt{(" "|\textbackslash n|\textbackslash t)+(" ")*}}
\begin{lstlisting}
{num_char++;
register int c;
escribe_palabra(salida," ");
while ((c = input())==' ');
if (c!=' ')
    unput(c);}
\end{lstlisting}
Detecta cuando hay espaciado entre palabras, y prosigue leyendo caracteres
hasta llegar a uno distinto de un espacio en blanco. Escribe un único espacio
en blanco en el archivo de salida.

\subsubsection*{\texttt{\textbackslash.|,|:}}
\begin{lstlisting}
{num_char++;escribe_palabra(salida,yytext);}
\end{lstlisting}
Detecta signos de puntuación en el texto y los transcribe, puesto que el resto
de reglas no lo hacen.

\newpage
\subsection{Ejecución del programa}

La función principal del programa consta de las siguientes sentencias:
\begin{lstlisting}
int main (int argc, char *argv[]) {
    if (argc!=3) {
        printf("No se ha proporcionado ningun fichero...\n");
        printf("Formato:\n\t./wikipedia archivo_entrada.txt archivo_salida.txt\n");
        exit(-1);
    }   
    else {
        yyin = fopen(argv[1],"rt");
        if (yyin==NULL) {
            printf("El fichero %s no se puede abrir\n", argv[1]);
            exit(-1);
        } 
        salida = fopen(argv[2],"w");
    }

    num_char = 0;
    lin_cor = 0;
    ref_elim = 0;
    p_tilde = 0;
    
    yylex();

    printf("\nLineas salida: %d\n",lin_cor);
    printf("Referencias eliminadas: %d\n",ref_elim);
    printf("Palabras con tilde: %d\n\n",p_tilde);

    exit(0);
}
\end{lstlisting}
El programa inicialmente comprueba que el número de argumentos de entrada sea
válido y asigna la entrada correspondiente a la herramienta \emph{Lex}. En 
caso de que no sean válidos o la asignación de la entrada no sea posible, 
aborta la ejecución del mismo. Tras ello, da los valores iniciales a las
variables explicadas anteriormente, y ejecuta la sentencia de procesamiento
del texto. Finalmente, imprime en pantalla las líneas del archivo de salida y
los números tanto de referencias eliminadas como de palabras únicas con tilde.

\newpage
\section{Archivos y uso}

El programa contiene un archivo tipo Makefile para la comodidad de uso del
mismo. Para compilar y ejecutar el programa, simplemente se debe escribir
la sentencia make dentro del directorio de trabajo en el que se encuentra el
archivo. Se pueden cambiar tanto el archivo de entrada como el archivo de
salida modificando las variables \texttt{IN} y \texttt{OUT} del Makefile o
invocando \texttt{make} y añadiendo nuevas asignaciones a esas variables.
\\

Además, junto con el programa hay 4 posibles textos de entrada que se
corresponden con secciones de artículos de Wikipedia en castellano. Estos
archivos se encuentran en el interior de la carpeta \textbf{doc}.


\end{document}